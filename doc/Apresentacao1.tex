\documentclass{beamer}

\usetheme[
  outer/progressbar=frametitle
]{metropolis}

\usepackage[portuguese]{babel}
\usepackage[utf8]{inputenc}

\hypersetup{colorlinks=true,urlcolor=blue,linkcolor=blue,citecolor=blue}

\title[Compiladores 2017.2]{Swimft - 1$^{a.}$ Apresentação}
\author[Abrev.]{Alunos: Beatriz e Vitor} 

\institute[UFF]{Universidade Federal Fluminense}

\date{Data: 15/05/2019} 

\begin{document}

% ---

\begin{frame}[plain]

\titlepage

\end{frame}

% ---

\begin{frame}{Linguagem escolhida: Swift}
\begin{description}
  \item[$\bullet$] Desenvolvida pela Apple e lançada em 2014
  \item[$\bullet$] Objective-C sem a corpulência do C, de sintaxe simples
\end{description}
\end{frame}

% ---

\begin{frame}{Suporte à construção de compiladores}
\begin{description}
  \item[$\bullet$] Linguagem com tipagem opcional
  \item[$\bullet$] Structs e funcs
  \item[$\bullet$] Error handling
\end{description}
\end{frame}

% ---

\begin{frame}{Prós e contras}
  \item[$\bullet$] 
  \item[$\bullet$] 
  \item[$\bullet$]
\end{frame}

%%%%%%%%

\title[Compiladores 2017.2]{Template para demais apresentações}
\author[Abrev.]{Autores} 

\institute[UFF]{Universidade Federal Fluminense}

\date{Data} 

\begin{document}

% ---

\begin{frame}[plain]

\titlepage

\end{frame}

% ---

\begin{frame}{Objetivos desta apresentação}
  \item[$\bullet$] 
  \item[$\bullet$] 
  \item[$\bullet$]
\end{frame}

% ---

\begin{frame}{O que foi feito}
  \item[$\bullet$] 
  \item[$\bullet$] 
  \item[$\bullet$]
\end{frame}

% ---

\begin{frame}{O que \emph{não} foi feito e porquê}
  \item[$\bullet$] 
  \item[$\bullet$] 
  \item[$\bullet$]
\end{frame}

% ---

\begin{frame}{Dúvidas}
  \item[$\bullet$] 
  \item[$\bullet$] 
  \item[$\bullet$]
\end{frame}

% ---

\begin{frame}{Avaliação da evolução do trabalho}
  \item[$\bullet$] 
  \item[$\bullet$] 
  \item[$\bullet$]
\end{frame}

\end{document}

