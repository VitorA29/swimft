\documentclass{beamer}

\usetheme[
  outer/progressbar=frametitle
]{metropolis}

\usepackage[portuguese]{babel}
\usepackage[utf8]{inputenc}

\hypersetup{colorlinks=true,urlcolor=blue,linkcolor=blue,citecolor=blue}

\title[Compiladores 2017.2]{Swimft}
\author[Abrev.]{Alunos: Beatriz e Vitor} 

\institute[UFF]{Universidade Federal Fluminense}

\date{Data: 17/05/2019} 

\begin{document}

% ---

\begin{frame}[plain]

\titlepage

\end{frame}

%%%%%%%%

\begin{frame}{Objetivos desta apresentação}
\begin{description}
  \item[$\bullet$] Apresentação de conclusão da primeira etapa do trabalho
  \item[$\bullet$] Dificuldades e aprendizado
\end{description}
\end{frame}

% ---

\begin{frame}{O que foi feito}
\begin{description}
  \item[$\bullet$] Ambientação com a linguagem
  \item[$\bullet$] Lexer 
  \item[$\bullet$] Parser para Imp-0
  \item[$\bullet$] Pi Framework: implementado compilador com operações aritméticas, booleanas e comandos
\end{description}
\end{frame}

% ---

\begin{frame}{O que \emph{não} foi feito e porquê}
\begin{description}
  \item[$\bullet$] Código completamente comentado
\end{description}
\end{frame}

% ---

\begin{frame}{Dificuldades}
\begin{description}
  \item[$\bullet$] Falta de documentação para instalação/configuração do swift no linux
  \item[$\bullet$] Falta de IDE free com suporte oficial
  \item[$\bullet$] LLVM extremamente técnico e complicado à primeira vista
\end{description}
\end{frame}

% ---

\begin{frame}{Avaliação da evolução do trabalho}
\begin{description}
  \item[$\bullet$] Terminar documento/código explicativo
  \item[$\bullet$] Here we go to Imp-1!
\end{description}
\end{frame}

\end{document}

